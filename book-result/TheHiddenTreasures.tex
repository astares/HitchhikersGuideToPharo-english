% -*- mode: LaTeX; -*-

\documentclass[a4paper,10pt,twoside]{book}
\usepackage[
        papersize={6.13in,9.21in},
        hmargin={.75in,.75in},
        vmargin={.75in,1in},
        ignoreheadfoot
]{geometry}
\input{../support/latex/common.tex}
% \input{../support/latex/commonLuaTex.tex}
\setboolean{lulu}{true}
%=================================================================
% Add the path for the figures of each chapter here:
\graphicspath{
        {../figures/}
        {../StartupPreferences/}
}
%=================================================================
\let\wholebook=\relax
\makeindex
\makeglossary
%=================================================================
\renewcommand{\nnbb}[2]{} % Disable editorial comments
%=================================================================
\begin{document}
\frontmatter
%=================================================================
\setcounter{page}{1}
\pagestyle{headings}
%=================================================================
\author{
  <Put the authors here>
}
\title{\Huge\bf {Hitchhikers Guide to Pharo} }
\isodate
\date{\emph{Version of \today}}
\maketitle
%=================================================================
\tableofcontents
% \listoffigures
% \listoftables

% \lstlistoflistings
\sloppy % To avoid LaTeX's annoying habit of letting lines stick over the margins!
\mainmatter

\chapter{ Zinc Tricks}
Zinc a really powerful HTTP framework developed by ven van Caekenberghe. Here are some hidden behavior. There are based on a real discussion between a user and Sven the author of Zinc. We let the conversation because it was great to gtet Zinc features exposed that way. 
\section{ Zinc and the Transcript}

% another try
% \printglossary
\bibliographystyle{jurabib}
\bibliography{scg}

\printindex

\end{document}

%=================================================================
%%% Local Variables:
%%% coding: utf-8
%%% mode: latex
%%% TeX-master: t
%%% TeX-PDF-mode: t
%%% ispell-local-dictionary: "english"
%%% End:
